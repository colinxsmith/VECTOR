\documentclass[12pt]{article}
\usepackage{setspace}
\usepackage{amsmath}
\oddsidemargin=0.0in
\evensidemargin=0.0in
\textwidth=6.5in
\headheight=0.0in
\topmargin=0.0in
\textheight=9.5in
\title{Setting Up An Optimisation To Minimise Loss}
%\date{16-12-2021}
\author{Colin M. Smith BITA Risk}
\begin{document}
\maketitle
\tableofcontents
\pagebreak
\doublespacing
\section{Introduction}
This document is for internal use only. It describes exactly how portfolio loss may be minimised
using linear programming. This is something that was developed at BITA; you can't look it up on the internet!
\section{Definition of Portfolio Gain and Loss}
Portfolio gain, $G$ and loss, $L$ are defined with respect to a target return $R$. Suppose the portfolio return
series at time $t$ is $r(t)$ then
\begin{eqnarray}
    G = \sum_t {\textbf{ max} }(r(t) - R,0)
\end{eqnarray}
\begin{eqnarray}
    L = \sum_t {\textbf{ max} }(R-r(t),0)
\end{eqnarray}
This means that in period $t$, if $R-r(t) > 0$ there is a loss, and if $R-r(t) < 0$ there is a gain.
If we define a loss period as one for which $R-r(t) > 0$,
\begin{eqnarray}
  {\rm Total\ Loss},\ L= \sum_{t\ \in\ \rm loss\ periods} R-r(t)
\end{eqnarray}
$G$ and $L$ are both non-negative and
\begin{eqnarray}
 {\rm   total\ portfolio\ return}= G - L + R
\end{eqnarray}

If there are $n$ assets in the portfolio and each asset $i$
has a weight $w_i$ and a return series $s_i (t)$ then
\begin{eqnarray}
    r(t) = \sum_{i=0}^{n-1} w_i s_i(t)
\end{eqnarray}
The expected return for asset $i$, $\alpha_i$ may be calculated from total\footnote{It's probably more traditional to define $\alpha$ as arithmetic mean, but if we do that here we should talk about average gain and loss rather than total.} 
return of $s_i(t)$.
\begin{eqnarray}
    \alpha_i =  \sum_{t=0}^{T-1} s_i(t)
\end{eqnarray}
where $T$ is the number of periods.
\section{Portfolio Optimisation}
We wish to maximise portfolio gain and minimise portfolio loss. The simplest way to achieve this is to try to
find the set of portfolio weights which maximise $G$-$\lambda L$ for unit wealth. Choosing a value for
$\lambda \ge 1$ and varying $w_i$
but keeping $\sum_i w_i = 1$ we try to maximise $G$-$\lambda L$. We can then assert that, for this loss tolerance $\lambda$, we have found the 
portfolio which has the largest gain for a loss of $L$ (or equivalently, the smallest loss for a gain of $G$).
There will be a unique set of portfolio weights for this maximisation as long as $\lambda\ge1$. ($G$-$\lambda L$
can be shown to be concave if $\lambda\ge1$). Note that because total return is $G-L+R$ we have;
\begin{eqnarray}
    \sum_i \alpha_i w_i = G-L+R
\end{eqnarray}
The portfolio problem we wish to solve is to maximise $G$-$\lambda L$ over portfolio weights $w_i$ for unit wealth.
This is equivalent to minimising the utility $U(w_1,w_2,$ $...w_n,t_1,t_2,...t_T)$ (because $R$ is constant). Our optimisation problem is;
\begin{align*}
    {\rm Minimise}\ & U(w_1,w_2,...w_n,t_1,t_2,...t_T)=\quad\cr \quad  & -\sum_i \alpha_i w_i + \lambda \left( \sum_{t_j\ \in\ \rm loss\ periods}( R-\sum_i w_i s_i(t_j) )\right)\cr
    {\rm Subject\ to}\cr & \sum_i w_i  = 1
\end{align*}
The loss periods depend on the portfolio weights, therefore $U(w_1,w_2,$ $...w_n,t_1,t_2,...t_T)$ is a non-linear function of portfolio weights.
\section{Solve an Equivalent Linear Problem}
The loss part of our utility function $U(w_1,w_2,$ $...w_n,t_1,t_2,...t_T)$ is made up of $T$ time intervals. Some of these
are loss periods, the others are gain periods and we have a variable $t_j$ for each time interval. If each loss period in $U$ had $t_j = {\rm loss}$ and each gain period had
$t_j=0$, we would have that the total loss is given by the sum of these variables $\Sigma_j t_j$. The problem is that at each step in the optimisation,
we don't know which periods have losses and which have gains. To overcome this we set up a linear relaxation, which puts constraints on the $t_j$ to make those in a gain period optimise to zero.
For this to work, the $t_j$ must satisfy
linking constraints to make $\sum s_i( t_j )w_i +t_j \ge R$. For a loss period this expresses that $t_j \ge$ period loss and for a gain period
$t_j \ge$ minus period gain. Since the optimisation is trying to minimise loss (because loss tolerance $\lambda > 0$), changing each $t_j$ relaxation variable, the optimal value 
for each $t_j$ in a loss period will be the period loss and the optimal value for $t_j$ in a gain period will be the lower bound, zero (this is $\ge$ - period gain).
By constraining each $t_j$ to be non-negative, we force the optimisation process to choose to constrain gain period 
$t_j$ to zero, so that the sum of all optimised $t_j$ variables is equal to total loss.

The relaxed linear program is (we denote the optimisation variables by $x_i$);
\begin{align*}
    {\rm Minimise}\ & \sum_{i=0}^{n+T-1} c_i x_i\cr
    &c_i=-\alpha_i &0\le i < n\cr
    &c_i =\lambda &n\le i < (n+T)\cr
    {\rm Subject\ to}\ &\cr& 0 \le x_i \le 1\ &0\le i < n\cr
    & 0 \le x_i \ &n\le i < n+T\cr
    & \sum_{i=0}^{n-1} x_i = 1\cr
    & R\ \le\sum_{i=0}^{n-1} x_i A_{ij} + x_j  &n\le j < n+T\cr
    & A_{ij} = s_i(t_j )\cr
\end{align*}
At convergence, the constraint $\sum_{i=0}^{n-1} w_i A_{ij} + w_j$ (total return for period $j$ plus its loss or gain) will take the lowest allowed value $R$ in a loss period or $(R+{\rm gain})$ in a gain period.
\end{document}